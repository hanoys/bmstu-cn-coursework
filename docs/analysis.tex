\chapter{Аналитическая часть}
\section{HTTP}
HTTP (HyperText Transfer Protocol) — это протокол прикладного уровня, используемый для передачи данных в распределенных информационных системах, таких как Всемирная паутина (World Wide Web). Он является основой взаимодействия между клиентами (например, браузерами) и серверами, обеспечивая обмен гипертекстом, мультимедийным контентом, данными и другими ресурсами~\cite{rfc9110}.

Основными характеристиками протокола HTTP являются:
\begin{itemize}
    \item Клиент-серверная архитектура: HTTP основывается на модели клиент-сервер: клиент (обычно браузер) отправляет запрос серверу, а сервер обрабатывает его и отправляет ответ.
    \item Без состояния (stateless): каждое взаимодействие между клиентом и сервером в HTTP независимое и не сохраняет информацию о предыдущих запросах. Это упрощает протокол, но требует дополнительных механизмов (например, cookies или session tokens) для реализации сессий. 
    \item Текстовый протокол: HTTP-запросы и ответы обычно представлены в текстовом формате, что делает их легкими для анализа и отладки.
    \item Простота использования: HTTP предоставляет четкую и понятную структуру запросов и ответов, что упрощает интеграцию в различные приложения.
\end{itemize}

HTTP-запрос состоит из следующих частей: 
\begin{enumerate}
    \item стартовая строка,
    \item заголовки,
    \item тело запроса.
\end{enumerate}

Токен метода, находящийся в стартовой строке, указывает действие, которое должно быть выполнено над ресурсом, идентифицированным с помощью URI. Токен метода может быть: GET, POST, PUT, DELETE, HEAD, OPTIONS, PATCH и другие~\cite{rfc2616}.

В листинге~\ref{lst:http.http} приведен пример HTTP запроса с использованием метода POST. 
\includelisting{http.http}{пример HTTP-запроса}


\section{Пул потоков}
Пул потоков (thread pool) — это механизм управления потоками в многопоточных приложениях, представляющий собой заранее созданный набор потоков, которые могут переиспользоваться для выполнения задач. Он позволяет эффективно управлять созданием, использованием и завершением потоков, минимизируя накладные расходы, связанные с их частым созданием и уничтожением~\cite{pool}.

Использование многопоточности является стандартным методом, используемым для обработки асинхронных запросов, с которыми обычно
работают веб-серверы и серверы баз данных. Для таких приложений многопоточность может повысить время ответа сервера, масштабируемость и
пропускную способность, а также улучшить взаимодействие между процессами. Из-за постоянного потока
сетевых запросов веб-серверы и сетевое промежуточное программное обеспечение особенно часто попадают в
категорию приложений, требующих большое количество операций ввода-вывода. 
Подавляющее большинство их времени тратится на ожидание ввода-вывода~\cite{pool2}. 

\section{Poll}
Системный вызов poll() предоставляет программе механизм мультиплексирования ввода/вывода по набору файловых дескрипторов.

В листинге~\ref{lst:poll.h} представлена сигнатура функции poll().
\includelisting{poll.h}{Сигнатура функции poll}

Параметры функции poll():
\begin{enumerate}
    \item fds --- массив элементов типа struct pollfd, который содержит файловые дескрипторы, которые будут отслеживаться с помощью функции poll();
    \item nfds --- количество элементов в массиве fds;
    \item timeout --- верхний предел времени, на которое будет блокироваться функция poll().
\end{enumerate}

В листинге~\ref{lst:pollfd.h} преведено определение структуры pollfd.
\includelisting{pollfd.h}{Определение структуры pollfd}

Поля events и revents структуры pollfd являются битовыми масками. 
Вызывающий объект инициализирует события, чтобы указать события, которые будут отслеживаться для файлового дескриптора fd. 
При возврате из функции poll() значение events устанавливается таким образом, 
чтобы указывать, какое из этих событий действительно произошло для данного файлового дескриптора~\cite{tlpi}.

\section*{Вывод}
\addcontentsline{toc}{section}{Вывод}
В данном разделе были рассмотрены основные принципы работы протокола HTTP, его характеристики, методы 
и пример запроса. Определены основные преимущества использования пула потоков для 
веб серверов, в том числе тех, что отдают статический контент. Рассмотрен системный 
вызов poll(), его сигнатура и принцип работы.
