\chapter*{ВВЕДЕНИЕ}
\addcontentsline{toc}{chapter}{ВВЕДЕНИЕ}
Статический сервер --- это сервер, который обслуживает статические файлы, такие как 
HTML, CSS, JavaScript, изображения, шрифты, видео и другие ресурсы, 
без какой-либо динамической обработки на стороне сервера. 
Его основная задача --- принимать запросы от клиента (например, веб-браузера) 
и возвращать запрашиваемые файлы, как они есть, из определённой директории.

Запросы всегда возвращают заранее подготовленные файлы, поэтому поведение статического сервера легко прогнозируется.
Статические серверы не поддерживают обработку бизнес-логики, взаимодействие с базами данных или создание страниц <<на лету>>. Для этих задач используются динамические серверы.

Цель работы --- реализовать статический сервер для отдачи файлов с диска
на основе пуля потоков и мультиплексора poll. 

Чтобы достичь поставленной цели, требуется решить следующие задачи:
\begin{itemize}
    \item провести анализ предметной области, 
    \item спроектировать алгоритмы работы сервера, 
    \item реализовать программу,
    \item провести нагрузочное тестирование.
\end{itemize}
