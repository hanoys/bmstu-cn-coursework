\chapter{Исследовательская часть}
\section{Технические характеристики}
Технические характеристики устройства, на котором выполнялись замеры по времени, представлены далее.

\begin{itemize}
	\item Процессор: Apple M2~\cite{M2}.
	\item Оперативная память: 16 ГБайт.
	\item Операционная система: MacOS Monterey 12.5.
\end{itemize}

При замерах времени ноутбук был включен в сеть электропитания и был нагружен только системными приложениями.

\section{Постановка исследования}
Целью данного исследования является проведение сравнение 
результатов прохождения нагрузочного тестирования с использованием утилиты 
Apache Benchmark (ab)~\cite{ab} с сервером Nginx~\cite{nginx}. 

Исследование направлено на изучение поведения разработанного сервера под нагрузкой и 
сравнение его производительности с Nginx.

В рамках тестирования был использован HTTP-запрос типа GET, направленный на URI, указывающий на файл index.html, который представляет собой простой статический ресурс.
Такое тестирование позволяет оценить скорость обработки запросов, устойчивость сервера при высоких нагрузках и эффективность его работы в типичных сценариях использования.

\section{Результаты исследования}
В таблице~\ref{tbl:time10} представлены результаты проведения нагрузочного тестирования 
разработанного сервера и сервера Nginx для 10 клиентов.
\begin{table}[H]
\caption{Время прохождения нагрузочного тестирования для 10 клиентов}
\label{tbl:time10}
\begin{tabular}{|l|l|l|}
\hline
Количество запросов &
  \begin{tabular}[c]{@{}l@{}}Время прохождения \\ тестирования\\ разработанного сервера, с\end{tabular} &
  \begin{tabular}[c]{@{}l@{}}Время прохождения \\ тестирования\\ сервера Nginx, с\end{tabular} \\ \hline
100  & 0.035 & 0.030 \\ \hline
500  & 0.086 & 0.082 \\ \hline
1000 & 0.148 & 0.127 \\ \hline
2000 & 0.291 & 0.215 \\ \hline
3000 & 0.416 & 0.274 \\ \hline
4000 & 0.537 & 0.405 \\ \hline
5000 & 0.678 & 0.453 \\ \hline
\end{tabular}
\end{table}

В таблице~\ref{tbl:time100} представлены результаты проведения нагрузочного тестирования 
разработанного сервера и сервера Nginx для 100 клиентов.
\begin{table}[H]
\caption{Время прохождения нагрузочного тестирования для 100 клиентов}
\label{tbl:time100}
\begin{tabular}{|l|l|l|}
\hline
Количество запросов &
  \begin{tabular}[c]{@{}l@{}}Время прохождения \\ тестирования\\ разработанного сервера, с\end{tabular} &
  \begin{tabular}[c]{@{}l@{}}Время прохождения \\ тестирования\\ сервера Nginx, с\end{tabular} \\ \hline
100  & 0.023 & 0.039\\ \hline
500  & 0.085 & 0.082\\ \hline
1000 & 0.171 & 0.129\\ \hline
2000 & 0.307 & 0.213\\ \hline
3000 & 0.383 & 0.305\\ \hline
4000 & 0.467 & 0.401\\ \hline
5000 & 0.625 & 0.494\\ \hline
\end{tabular}
\end{table}

\section*{Вывод}
\addcontentsline{toc}{section}{Вывод}
Из приведенных таблиц и графиков видно, что в среднем разработанный сервер работает на 30\% медленее 
сервера Nginx. При малом количество клиентов и запросов разница производительности несущественна, 
однако при числе запросов равному 5000 Nginx работает в 1.5 раза быстрее. 
